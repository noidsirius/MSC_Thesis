
% -------------------------------------------------------
%  Abstract
% -------------------------------------------------------


\pagestyle{empty}

\شروع{وسط‌چین}
\مهم{چکیده}
\پایان{وسط‌چین}
\بدون‌تورفتگی
مسائل بهینه‌سازی هندسی از دیرباز در علوم کامپیوتر مورد بررسی قرار گرفته‌اند و کاربر بسیار زیادی در حوزه‌های مختلف دارند. از مهم‌ترین این مسائل می‌توان به  مسائل کوچک‌ترین کره‌ی محیطی و $k$-مرکز اشاره کرد. با افزایش سرعت تولید حجم داده پژوهش‌های اخیر روی مدل‌های داده‌های حجیم مانند جویبار داده و پنجره‌ی لغزان متمرکز شده است. تمرکز اصلی این پایان‌نامه روی حل مسائل بهینه‌سازی هندسی (به طور خاص $k$-مرکز هندسی) در مدل پنجره‌ی لغزان است. در مدل پنجره‌ی لغزان به دنبال پاسخ‌گویی به پرسش روی $N$ نقطه‌ی آخر ورودی هستیم. از مشکلات این روش می‌توان به عدم امکان نگه‌داری تمام نقاط اشاره کرد.

در این پایان‌نامه، مسائل کوچک‌ترین کره‌ی محیطی، $2$-مرکز و $k$–مرکز هندسی در مدل پنجره‌ی لغزان مورد بررسی قرار می‌گیرد. برای مسئله‌ی کوچک‌ترین کره‌ی محیطی در فضای $d$-بعدی الگوریتم $(1+\eps)$-تقریب با حافظه‌ی $ \cO(\log R \frac{\sqrt{d}}{\eps^{d+1}})$ و زمان پردازش نقاط  $ \cO((d+1)! \log R \frac{\sqrt{d}}{\eps^{d+1}})$ ارائه می‌دهیم که پارامتر $R$ برابر نسبت بیش‌ترین پاسخ به کوچک‌ترین فاصله‌ی بین دو نقطه است.  این الگوریتم اولین الگوریتم ارائه شده برای فضای بیش از دو بعد است. سپس برای مسئله‌ی $2$-مرکز دوبعدی الگوریتم $(1+\eps)$-تقریب ارائه می‌شود که الگوریتم قبلی با ضریب تقریب $4+\eps$ را بهبود می‌دهد. 
در پایان مسئله‌ی $k$-مرکز به وسیله‌ی یک الگوریتم $(2+\eps)$-تقریب در مدل پنجره‌ی لغزان تقریب زده می‌شود که الگوریتم قبلی با ضریب تقریب $6+\eps$ را بهبود می‌دهد. حافظه‌ی مصرفی تمام الگوریتم‌ها از مرتبه‌ی چندجمله‌ای نسبت به $d$ و $\log R$ و  {$\frac{1}{\eps^d}$ است. لازم به ذکر است که کران پایین $\log R$ برای حافظه‌ی الگوریتم‌هایی از پنجره‌ی لغزان ثابت شده است.


\پرش‌بلند
\بدون‌تورفتگی \مهم{کلیدواژه‌ها}: 
بهینه‌سازی هندسی، پنجره‌ی لغزان، الگوریتم‌های تقریبی، داده‌های حجیم
\صفحه‌جدید
