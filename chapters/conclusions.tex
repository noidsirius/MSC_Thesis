
\فصل{نتیجه‌گیری}

در این پایان‌نامه مسائل مختلفی از بهینه‌سازی هندسی در مدل پنجره‌ی لغزان بررسی شد. این مسائل شامل قطر، کره‌ی محصور کمینه، $2$-مرکز دوبعدی و $k$-مرکز هندسی بود. این مسائل در مدل ایستا سابقه‌‌ی بسیار طولانی و کاربرد بسیار زیادی دارند. با توجه به افزایش سرعت تولید و امکان جمع‌آوری داده‌ها ضرورت حل این مسائل در مدل‌های داده‌های حجیم (مثل پنجره‌ی لغزان) احساس می‌شود. 

در این پژوهش به جای تمرکز روی تعدادی مسئله‌ی خاص، خانواده‌ای از مسئله‌های بهینه‌سازی هندسی را معرفی کردیم و چارچوبی برای حل آن‌ها در مدل پنجره‌ی لغزان ارائه کردیم. مسائل $C$-پوشا شامل مسائلی می‌شوند که به دنبال بیشینه‌کردن پارامتری در نقاط هستیم که به نحوی به قطر آن‌ها مرتبط باشد. چارچوب ارائه‌شده با ورودی گرفتن یک زیرالگوریتم حل دقیق مسئله، تقریبی معادل با $1+\eps$ از مسئله در مدل پنجره‌ی لغزان ارائه می‌دهد. به صورت دقیق‌تر برای مسئله‌ی کره‌ی محصور پوشا در فضای $d$-بعدی الگوریتم $(1+\eps)$-تقریب با حافظه‌ی $ \cO(\log R \frac{\sqrt{d}}{\eps^{d+1}})$ و زمان پردازش نقاط  $ \cO(\log R \frac{\sqrt{d}}{\eps^{d+1}})$ ارائه دادیم. تنها نمونه‌ی مشابه این الگوریتم در فضای دوبعدی وجود دارد که حافظه‌ی بیش‌تری استفاده می‌کند \مرجع{chan2006geometric}.

در ادامه مسئله‌ی $2$-مرکز هندسی در صفحه را مورد بررسی قرار دادیم. با استفاده از چارچوب ذکرشده الگوریتم $(1+\eps)$-تقریب با حافظه‌ی مصرفی $ \cO(\log R \frac{1}{\eps^{3}})$ و زمان پردازش $ \cO(\log R \frac{1}{\eps^3} poylog(\frac{1}{\eps}))$ برای حل این مسئله ارائه دادیم. این الگوریتم از نظر ضریب تقریب بهبود بسیار زیادی نسبت به بهترین نمونه‌ی قبلی ($(4+\eps)$-تقریب در فضای متریک) ایجاد کرده است\مرجع{DBLP:conf/icalp/Cohen-AddadSS16}.

و در بخش آخر پژوهش به مسئله‌ی $k$-مرکز پرداختیم. با توجه به این که مسئله به طور عمومی جزو مسائل $C$-پوشا قرار نمی‌گرفت و همین‌طور الگوریتم دقیق آن در مدل ایستا ناکارآمد بود روش جدیدی ارائه کردیم. الگوریتم ما دارای ضریب تقریب $2+\eps$، حافظه‌ی مصرفی $ \cO(k \log R \frac{d}{\eps^{d+1}})$ و زمان پردازش ورود نقطه‌ی جدید از مرتبه‌ی $ \cO(k^2 \log R \frac{d}{\eps^{d+1}})$ است. بهترین و تنها پژوهش انجام گرفته در این حوزه ضریب تقریب $6+\eps$ دارد که در مقایسه با الگوریتم ما ضریب تقریب خیلی بالاتری (۳ برابر) است. \مرجع{DBLP:conf/icalp/Cohen-AddadSS16}

\قسمت{کارهای آتی}

یک ویژگی مهم نتیجه‌ی پژوهش ما چارچوب حل مسائل برای مدل پنجره‌ی لغزان است. به عبارت دیگر هر چقدر بتوان مسائل $C$-پوشا را در مدل ایستا سریع‌تر حل کرد، سرعت حل آن‌ها در مدل پنجره‌ی لغزان نیز افزایش می‌یابد. علاوه بر این با بررسی دقیق‌تر حالت‌های خاص مسائل، مثل محدودکردن ابعاد فضا یا پارامترهای ورودی،  می‌توان راه حل‌های سریع‌تری پیدا کرد (در همین پایان‌نامه الگوریتم $2$-مرکز دوبعدی بررسی شد).

از طرف دیگر میزان حافظه‌ی مصرفی چارچوب هم قابل بهبود است. در این پژوهش برای کاهش حافظه از ایده‌ی توری استفاده کردیم. تمرکز اصلی ما در این روش محدودکردن خطای تقریب برای مقادیر خاصی از اندازه‌ی پاسخ مسئله بود تا بتوان از ایده‌ی موازی‌سازی استفاده کرد. در صورتی که روش‌های دیگری برای ارضای این هدف ارائه شود امکان کاهش خیلی بیش‌تر حافظه خواهد بود.

و در پایان می‌توان به حالت‌های دیگری از مسئله‌ی $k$-مرکز اشاره کرد که امکان حل آن در این بستر وجود دارد. یکی از این نمونه‌ها مسئله‌ی $k$-مرکز با داده‌ی پرت است که در مدل ایستا و جویبار داده بسیار مورد بررسی قرار گرفته است اما تا به حال در مدل پنجره‌ی لغزان هیچ الگوریتمی برای تقریب آن ارائه نشده است.


