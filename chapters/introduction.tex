
\فصل{مقدمه}


مسائل بهینه‌سازی هندسی \پاورقی{Geometry Optimization} با این که سابقه‌ی طولانی در علوم محاسباتی دارند اما هنوز کهنه نشده‌اند و با معرفی هر مدل نگه‌داری داده‌ی جدید، نیاز به بررسی و تحقیق روی آن‌ها دوباره احساس می‌شود. از طرف دیگر با افزایش کاربردهای مکان-محور \پاورقی{Location-Based} در صنعت لزوم بهبود این الگوریتم‌ها ضروری‌تر به نظر می‌رسد. مسائلی از قبیل پوش محدب \پاورقی{Covex Hull }، پیداکردن قطر \پاورقی{Diameter} یا عرض \پاورقی{Width}، کوچک‌ترین کره‌ی محیطی \پاورقی{ Minimum Enclosing Ball} یا شناسایی $k$-مرکز \پاورقی{$k$-centers}  همگی در دسته‌ی مسائل بهینه‌سازی هندسی قرار می‌گیرند. مسئله‌ی $k$-مرکز یک رویکرد برای حل مسئله‌ی خوشه‌بندی است که خود یکی از مهم‌ترین مسائل داده‌کاوی \پاورقی{ Data Mining} به شمار می‌آید. 

از طرف دیگر با افزایش حجم اطلاعات در مسائل دنیای واقعی، مدل‌های نگه‌داری داده‌ای معرفی شدند که برای محاسبات محدودیت‌هایی در اندازه‌ی حافظه و نحوه‌ی دسترسی به آن را اعمال می‌کنند. به عنوان مثال مدل جویبار داده \پاورقی{Data Stream} و پنجره‌ی لغزان \پاورقی{ Sliding Window}  را می‌توان نام برد. در اولی همان‌طور که از نامش بر می‌آید داده‌ها در یک جویبار یک به یک وارد می‌شوند و تنها یک (یا تعداد مشخص و محدودی) بار می‌توانیم داده‌ها را ببینیم و حافظه‌ای که می‌توانیم استفاده کنیم از مرتبه‌ی زیرخطی \پاورقی{Sublinear} است. مدل پنجره‌ی لغزان از جویبار داده مشتق شده است با این ویژگی که می‌خواهد محاسبات تنها داده‌هایی  را لحاظ کند که اخیرا وارد شده‌اند (مثلا $N$ داده‌ی آخر). با توجه به تعریف این مدل‌های داده می‌توان به این نتیجه رسید که برای حل مسائل کلاسیک در این مدل‌ها نیاز به نگاه از زاویه‌ای دیگر است. از طرف دیگر به خاطر ذات این مدل‌ها که حافظه‌ی زیرخطی دارند بیش‌تر مسائل را نمی‌توان به صورت دقیق در این مدل حل کرد و معمولا به صورت تقریبی حل می‌شوند.

در این پژوهش، با تمرکز روی مسائل بهینه‌سازی هندسی در مدل پنجره‌ی لغزان، مسئله‌ی $k$-مرکز در فضای هندسی با ابعاد کوچک مورد بررسی قرار گرفته است که بهبود قابل توجهی در ضریب تقریب این مسئله به دست آمده است.
در بخش بعدی، تعریف رسمی\پاورقی{Formal} از مسائلی که در این پایان‌نامه مورد بررسی قرار می‌گیرند را بیان نموده و در مورد هر کدام توضیح مختصری می‌دهیم.

\قسمت{تعریف مسئله}

تعریف دقیق‌تر مسئله‌ی $k$-مرکز در زیر آمده است:
\شروع{مسئله}[$k$-مرکز]
مجموعه‌ی $P$ شامل تعدادی نقطه است. فاصله‌ی این نقاط به وسیله‌ی تابع فاصله‌ی $dis$، که از نامساوی مثلثی\پاورقی{Triangle Inequality} پیروی می‌کند به دست می‌آید.
زیرمجموعه‌ی $k$ عضوی $T \subseteq P$ را طوری انتخاب کنید که عبارت زیر را کمینه کند:
\شروع{equation}
\max_{p \in P} \{ \min_{t \in T} dis(p, t) \}
\پایان{equation}
\پایان{مسئله}

در صورتی که تابع فاصله‌ی $dis$ در فضای اقلیدسی باشد به این مسئله $k$-مرکز هندسی می‌گوییم. یک تفاوت ساختاری فضای متریک با هندسی در این است که در فضای هندسی می‌توان فاصله‌ی نقاطی که بین نقاط ورودی نیست را با دیگر نقاط به دست آورد اما در فضای متریک تنها فاصله‌ی هر دو نقطه‌ای که در ورودی آمده است را داریم.

مسئله‌ی $k$-مرکز در مدل جویبار داده توجه زیادی را به خود جلب کرده است و مورد بررسی ‌های زیادی قرار گرفته است. در این مدل ابتدا تمام نقاط در دسترس نیستند، بلکه یکی پس از دیگری وارد می‌شوند. هم‌چنین ترتیب ورود نقاط نامشخص است. 
علاوه بر این محدودیت مدل جویبار داده دارای محدودیت حافظه است، به‌طوری‌که امکان نگه‌داری تمام نقاط در حافظه  وجود ندارد و معمولا باید مرتبه‌ی حافظه‌ای کم‌تر از مرتبه حافظه‌ی \مهم{خطی}\پاورقی{Linear} (یا همان زیرخطی) متناسب با تعداد نقاط استفاده نمود.

مدلی که ما در این پژوهش بر روی آن تمرکز داریم مدل پنجره‌ی لغزان است که از مدل جویبار داده تک‌گذره\پاورقی{Single pass} ~\مرجع{aggarwal2007data} مشتق شده‌ است.
یعنی تنها یک بار می‌توان از ابتدا تا انتهای داده‌ها را بررسی کرد و پس از عبور از یک داده، اگر آن داده در حافظه ذخیره نشده باشد، دیگر نمی‌توانیم به آن دسترسی داشته باشیم. علاوه بر این، در هر لحظه باید بتوان به پرسمان (برای $N$ نقطه‌ی اخیر) پاسخ داد.

یکی از دغدغه‌هایی که در مسائل جویبار داده  و پنجره‌ی لغزان وجود دارد، عدم امکان دسترسی به تمام نقاط است.
به عبارت دیگر نه می‌توانیم به تمام داده‌هایی که تا الان آمده‌اند دسترسی داشته باشیم و نه می‌توانیم راجع به داده‌هایی که هنوز وارد نشدند نظری بدهیم. در مدل پنجره‌ی لغزان حتی باید به این توجه کنیم که داده‌هایی که در حافظه ذخیره کرده‌ایم ممکن است منقضی بشوند (یا از پنجره‌ خارج شوند). یعنی نمی‌توانیم به هیچ کدام از نقاطی که تا به حال ذخیره کردیم اعتماد کنیم. چون تا پایان نمی‌توانند معتبر باشند.
\شروع{تعریف}
\برچسب{تعریف:LP}
\مهم{$L_p$ متریک}
به ازای دو نقطه‌ی $d$-بعدی $s$ و $q$، فاصله‌ی $s$ و $q$ در متریک $L_p$ برابر با
$$d(p, q) = \sqrt[p]{\sum_{i=1}^{d} (s_i - q_i) ^ p}$$
است. که $s_i$ و $q_i$ برابر مختصات بعد $i$ام نقاط $s$ و $q$ است. 
\پایان{تعریف}
لازم به ذکر است که $L_2$ متریک همان تابع فاصله در فضای اقلیدسی است. مسئله‌ی $k$-مرکز، معمولاً تنها برای $L_p$-متریک مطرح می‌شود (زیرا نیاز به دانستن فاصله‌ی هر دو نقطه‌ای است). در حالت دیگر باید مجموعه‌ای از تمام نقاط فضا به انضمام فاصله‌هایشان را داشته باشیم.
%زیرا مرکز دسته‌ها ممکن است در هر نقطه از فضا قرار بگیرد و ما نیاز داریم که فاصله‌ی آن را از تمام نقاط بدانیم.
%نمونه‌ای از مسئله‌ی $1$-مرکز در حالت پیوسته و گسسته، در شکل \رجوع{شکل:دومرکزپیوسته} نشان داده شده است.
%
%
%\شروع{شکل}[ht]
%\centerimg{continious-k-center}{10cm}
%\شرح{نمونه‌ای از‌مسئله‌ی ۲-مرکز در حالت پیوسته}
%\برچسب{شکل:دومرکزپیوسته}
%\پایان{شکل}

تعریف دقیق گونه‌ی پنجره‌ی لغزان مسئله‌ی $k$‌-مرکز، در زیر آمده است:
\شروع{مسئله}
\مهم{($k$-مرکز در مدل پنجره‌ی لغزان)} دنباله‌ی $U$ از نقاط فضای $d$-بعدی داده شده است. $P$ را $N$ نقطه‌ی آخر $U$ می‌نامیم.
زیرمجموعه $S \subseteq P$ با اندازه‌ی $k$ را انتخاب کنید به‌طوری‌که عبارت زیر کمینه شود:
\شروع{equation}
\max_{u \in P} \{ \min_{s \in S} L_p(u, s) \}
\پایان{equation}
\پایان{مسئله}
با مطالعه‌ی پژوهش‌های انجام‌شده در حوزه‌های الگوریتم‌های تقریبی بهینه‌سازی هندسی و مدل‌های داده‌حجیم (مثل جویبار داده و پنجره‌ی لغزان) تصمیم گرفتیم تمرکز این پژوهش را روی $k$-مرکز هندسی در ابعاد پایین بگذاریم.

\قسمت{اهمیت موضوع}

به علت افزایش سریع حجم و تولید داده‌ها دیگر امکان پردازش و دسترسی آزادانه به تمام داده‌ها وجود ندارد. به همین دلیل مسئله‌های مدل جویبار داده در سالیان اخیر بسیار مورد توجه قرار گرفته‌اند. اگر از زاویه‌ی دیگری به سرعت بالای تولید اطلاعات نگاه کنیم، متوجه می‌شویم که نه تنها دسترسی دلخواه به تمامی داده‌ها نداریم بلکه علاقه‌ای نیز به داده‌های بسیار قدیمی وجود ندارد. برای روشن‌تر شدن این حوزه دو مثال از دنیای واقعی می‌زنیم. 


\شروع{مثال}
\برچسب{مثال:روتر}
یک مسیریاب \پاورقی{Router} را در نظر بگیرید که بسته‌های \پاورقی{Packet} شبکه را از گره \پاورقی{Node} مبدا می‌گیرد و به گره مقصد تحویل می‌دهد. به حجم ارتباطات شبکه روز به روز افزوده می‌شود و امکان نگه‌داری (حتی داده‌های ضروری) بسته‌ها وجود ندارد. یک مسئله در مسیریاب‌ها شناسایی پربازدیدترین مقصدها است. مثلا می‌خواهیم بدانیم از شبکه‌ی داخلی \لر{sharif.ir} چه آدرسی بیش‌ترین بازدید را داشته است. اگر دامنه‌ی محاسبات را تمامی بسته‌هایی که از مسیریاب اصلی دانشگاه شریف ،از بدو شروع به کار آن، گذشته است در نظر بگیریم به پربازدیدترین آدرس در تمام سالیانی که این مسیریاب کار می‌کرده خواهیم رسید. اما این که بدانیم در هفته یا ماه گذشته چه آدرسی بیش‌ترین بازدید را داشته است بسیار ارزشمندتر است چرا که داده‌های سالیان گذشته تاثیری روی کاربرد فعلی مسیریاب نخواهد داشت.
\پایان{مثال}
\شروع{مثال}
\برچسب{مثال:اینستاگرام}
شبکه‌ی اجتماعی اینستاگرام \پاورقی{Instagram} قابلیتی به نام داستان \پاورقی{Story} دارد که هر کاربر می‌تواند عکس یا فیلمی کوتاه را به صورت داستان به دیگر دنبال‌کنندگانش نشان دهد. هر داستان تا ۲۴ ساعت به دیگر کاربران نمایش داده می‌شود و پس از آن حذف می‌شود. هر داستان می‌تواند برچسب مکان داشته باشد که نشان‌دهنده‌ی جایی است که آن داستان رخ داده است. اینستاگرام به هر کاربر علاوه بر داستان‌های افرادی که دنبال می‌کنند داستان‌هایی که برچسب مکان آن‌ها نزدیک به مکان فعلی کاربر هست را نیز نمایش می‌دهد.
 \پایان{مثال}
در مثال \رجوع{مثال:روتر} متوجه حجم بالای داده‌ها و عدم امکان دسترسی آزاد به تمامی آن‌ها می‌شویم. علاوه بر این در مثال \رجوع{مثال:اینستاگرام} دیده می‌شود تمرکز برنامه‌ها روی داده‌های اخیر (مثلا ۲۴ ساعت گذشته) است و حتی امکان دسترسی به داده‌های قدیمی نیز وجود ندارد.
این‌جا ضرورت مدل‌ داده‌ای احساس می‌شود که نه تنها بتواند دسترسی به این حجم بزرگ اطلاعات را محدود و کنترل کند بلکه مسئله را به سمت استفاده از داده‌های اخیر متمرکز کند. مدل پنجره‌ی لغزان با دارابودن خواص جویبار داده محدودیت‌هایی را اضافه کرده است که باعث تمرکز بر داده‌های جدید شده است. به همین دلیل در این پژوهش مسئله‌ی اصلی بر روی این مدل داده معطوف شده است.
در مثال \رجوع{مثال:روتر} اگر دامنه‌ی محاسبات را تمامی بسته‌ها در نظر بگیریم می‌توان از مدل جویبار داده استفاده کرد. اما اگر مجموعه‌ی مورد نظر را به بسته‌های یک ماه اخیر کاهش دهیم یا در مثال \رجوع{مثال:اینستاگرام} بهتر است از مدل پنجره‌ی لغزان استفاده کرد که علاوه بر این که محدودیت دسترسی به داده‌ها را ارضا می‌کند بلکه تمرکزش روی داده‌های اخیر است.

در حوزه‌ی بهینه‌سازی هندسی، مسئله‌ی $k$-مرکز و گونه‌های آن از جمله کاربردی‌ترین و متداول‌ترین مسائل به شمار می‌آیند. کاربرد این مسئله در مباحث داده‌کاوی بسیار جا افتاده است و یکی از رایج‌ترین الگوریتم‌های مورد استفاده برای خوشه‌بندی محسوب می‌شود. از طرف دیگر کاربردهای زیادی در دنیای واقعی دارد.
 به عنوان مثال فرض کنید در مثال \رجوع{مثال:اینستاگرام}، اینستاگرام می‌خواهد به جای نمایش کل داستان‌های شهر به هر کاربر، داستان‌هایی را نشان دهد که در فاصله‌ی نزدیک‌تری به وی هستند. یک روش برای مدل‌سازی این مسئله استفاده از مسئله‌ی $k$-مرکز است تا کاربران را به دسته‌هایی تقسیم کند که فاصله‌‌یشان خیلی کم است. به طور دقیق‌تر به خاطر ماهیت داستان (که پس از ۲۴ ساعت از بین می‌رود) بهتر است مسئله‌ی $k$-مرکز در فضای اقلیدسی دوبعدی پنجره‌ی لغزان به عنوان مدل در نظر گرفته شود.

  به دلایل بالا این پژوهش روی مسائل بهینه‌سازی هندسی (به طور خاص $k$-مرکز هندسی) در مدل پنجره‌ی لغزان متمرکز شده است.
\قسمت{ادبیات موضوع}
مسئله‌ی $k$-مرکز یکی از مهم‌ترین مسائل بهینه‌سازی هندسی است که در خانواده‌ی مسائل $NP$-سخت قرار دارد. به شرط $P \neq NP$، هیچ الگوریتم دقیقی برای حل این مسئله در زمان چندجمله‌ای حتی در مدل ایستا هم وجود ندارد. در نتیجه در بیش‌تر مواقع مجبور هستیم از الگوریتم‌های تقریبی \پاورقی{Approximation algorithm}  استفاده کنیم.
علاوه بر این مسئله، مسائل ساده‌تری مانند محاسبه‌ی قطر یا عرض و یا کوچک‌ترین کره‌ی محیطی وجود دارد. این مسائل  از دیرباز در مدل ایستا بسیار مورد بررسی قرار گرفته‌اند. اکنون این مسائل بیش‌تر در مدل‌های داده حجیم مورد بررسی قرار می‌گیرند.

برای مسئله‌ی $k$-مرکز، الگوریتم‌های تقریبی معروفی وجود دارد که به یکی از ساده‌ترین آن‌ها اشاره می‌کنیم.
این الگوریتم از رویکرد حریصانه \پاورقی{Greedy} استفاده می‌کند.  ابتدا یک نقطه‌ی دلخواه را به عنوان مرکز در نظر می‌گیرد سپس در هر مرحله نقطه‌ای را به عنوان مرکز انتخاب می‌کند که از بقیه‌ی مراکز بیش‌ترین فاصله را داشته باشد.\مرجع{megiddo1984complexity}.
این الگوریتم، الگوریتم تقریبی با ضریب تقریب 2 ارائه می‌دهد.
همچنین کران پایین تقریب این مسئله مشخص شده‌است. بهتر از ضریب تقریب $2$  برای مسئله‌ی $k$-مرکز در حالت کلی نمی‌توان الگوریتمی یافت به شرط آن که $P \neq NP$ باشد.

برای مسئله‌ی $k$-مرکز در حالت جویبار داده برای ابعاد بالا، بهترین الگوریتم موجود ضریب تقریب $2 + \epsilon$ دارد \مرجع{mccutchen2008streaming, guha2009tight, ahn2014computing} و ثابت می‌شود الگوریتمی با ضریب تقریب بهتر از $2$ نمی‌توان ارائه داد.

برای مسئله‌ی $k$-مرکز مدل پنجره‌ی لغزان نیز، بهترین الگوریتم ارائه شده، الگوریتمی با ضریب تقریب $6$ است که  در فضای متریک ارائه شده است \مرجع{DBLP:conf/icalp/Cohen-AddadSS16}. 

برای $k$های کوچک به خصوص، $k =1, 2$، الگوریتم‌های بهتری ارائه شده است. بهترین الگوریتم ارائه شده برای مسئله‌ی $1$-مرکز در حالت جویبار داده برای ابعاد بالا، دارای ضریب تقریب $1.22$ است و کران پایین $\frac{1 + \sqrt{2}}{2}$ نیز برای این مسئله اثبات شده است \مرجع{agarwal2010streaming, chan2014streaming}. برای مسئله $2$-مرکز در مدل پنجره‌ی لغزان فضای متریک، راه‌حلی با ضریب تقریب $4$ ارائه شده است \مرجع{DBLP:conf/icalp/Cohen-AddadSS16}. برای مسئله‌ی $1$-مرکز مدل پنجره‌ی لغزان در فضای دوبعدی، الگوریتمی با ضریب تقریب $1+ \epsilon$ است که به وسیله‌ی ارائه‌ی یک $\epsilon$-هسته به دست آمده است. \مرجع{chan2006geometric}.

\قسمت{اهداف تحقیق}

در این پایان‌نامه سعی شده است تا مسائلی از حوزه‌ی بهینه‌سازی هندسی (با تمرکز روی $k$-مرکز) شناسایی شود تا در مدل پنجره‌ی لغزان به طور کارآمدی حل شوند و اگر نتایج قبلی در برخی موارد وجود داشته از جنبه‌های مختلف بهبود دهد.

مسئله‌ی اولی که بررسی شده است، محاسبه‌ی قطر نقاط در مدل پنجره‌ی لغزان است. این مسئله معادل شناسایی بیش‌ترین فاصله بین هر دو نقطه در یک پنجره است. الگوریتم بهینه‌ی $(1+\epsilon)$-تقریب این مسئله در فضای دوبعدی در  \مرجع{chan2006geometric} آمده است و برای فضای متریک نیز روش $3$-تقریب وجود دارد که با در نظرگرفتن محدودیت حافظه این ضریب تقریب بهینه است. هدف ما از بررسی این مسئله شناسایی روشی کلی برای حل مسائل بهینه‌سازی هندسی در مدل پنجره‌ی لغزان است. روش ارائه‌شده توسط ما ضریب تقریب $(1+\epsilon)$ دارد.

دومین مسئله‌ی مورد مطالعه، معرفی بستری برای حل دسته‌ای از مسائل بهینه‌سازی هندسی (شامل قطر و $k$-مرکز) است که در مدل پنجره‌ی لغزان قابلیت حل‌شدن با حافظه‌ی مناسب دارند. بستری که ارائه شده است با استفاده از الگوریتم تقریبی یا دقیق مدل ایستا می‌تواند همان مسئله را در مدل پنجره‌ی لغزان با تقریب $(B+\epsilon)$ حل کند ($B$ ضریب تقریب حل مسئله در مدل ایستا است). 

در مسئله‌ی سوم به طور اختصاصی روی $k$-مرکز در فضای دوبعدی تمرکز کردیم. در تلاش اول برای مسئله‌ی $1$-مرکز روشی با تقریب $(1+\epsilon)$ و حافظه‌ی $O(\frac{1}{\epsilon} lg R)$ و زمان به‌روز‌رسانی $O(\frac{1}{\epsilon})$ ارائه دادیم. سپس روشی برای $2$-مرکز به دست آوردیم و در پایان برای حل مسئله‌ی $k$-مرکز با ضریب تقریب $(2+\epsilon)$، حافظه‌ی  $O(\frac{1}{\epsilon} lg R)$ و زمان به روز‌رسانی $O(\frac{1}{\epsilon})$ به دست آوردیم که از بهترین روش موجود که $6$-تفریب است بهبود قابل توجهی پیدا کرده است.


\قسمت{ساختار پایان‌نامه}
این پایان‌نامه از پنج فصل تشکیل شده‌است که به شرح آن می‌پردازیم. فصل اول (همین فصل) راجع به مقدمات پژوهش توضیحاتی 
ارائه کرد. در فصل دوم مفاهیم و تعاریف مرتبط با موضوعات پژوهشی این پایان‌نامه بیان می‌کنیم. در این فصل تلاش بر این بوده تا با توضیحات کلی و ساده راجع به فضای پژوهش اطلاعات ضروری و مفید منتقل شود.
فصل سوم این پایان‌نامه شامل مطالعه و در برخی موارد عمیق‌شدن در پژوهش‌های پیشین انجام شده‌ی مرتبط با موضوع این پایان‌نامه خواهد بود.
این فصل در سه بخش تنظیم گردیده است.
در بخش اول، مسائل $k$-مرکز و قطر و عرض در حالت ایستا مورد بررسی قرار می‌گیرد.
در بخش دوم، مدل پنجره‌ی لغزان مسائل قطر و عرض و روش‌های مرسوم حل این مسائل مورد بررسی قرار می‌گیرد. در نهایت، در بخش سوم، مسئله‌ی $k$-مرکز در مدل پنجره‌ی لغزان مورد بررسی قرار می‌گیرد.

در فصل چهارم، نتایج جدیدی که در این پژوهشبه آن دست پیدا کرده‌ایم ارائه می‌شود.
این نتایج شامل چارچوب ارائه‌شده برای تقریب $(1+\eps)$ دسته‌ای از مسائل بهینه‌سازی هندسی در مدل پنجره‌ی لغزان و ارائه‌ی روش $(2+\epsilon)$-تقریب برای مسئله‌ی $k$-مرکز  است.

و در نهایت فصل پنجم به جمع‌بندی اختصاص دارد. در ابتدا خلاصه‌ای از نتایج به دست‌آمده و در قسمت دوم کارهای آینده که در طول این پژوهش به آن فکر کرده‌ایم وجود دارد.
